% --------------------------------------------------
% Resumo e abstract (obrigatórios)
% --------------------------------------------------
\resumo{%
O mercado financeiro atrai a atenção de muitos há séculos, abrangendo ativos variados como arroz, ouro e ações. A flutuação dos preços pode indicar eventos importantes, e prever tais eventos pode determinar o sucesso ou fracasso de um negócio. Diversas técnicas de previsão, como a análise técnica e fundamentalista, surgiram ao longo dos anos. Recentemente, redes neurais, especialmente Long Short-Term Memory (LSTM), têm identificado padrões em séries temporais antes despercebidos. Paralelamente, mercados voláteis como o de criptomoedas emergiram, especialmente após o surgimento do Bitcoin e da tecnologia Blockchain em 2008, oferecendo alternativas para combater a inflação e o controle centralizado. Este estudo visa analisar e comparar algoritmos, em uma base de dados real, que buscam predizer o comportamento dos ativos em mercados voláteis, principalmente no contexto do Bitcoin. Foram utilizadas diferentes abordagens baseadas em descobertas em artigos recentes, proporcionando também variabilidade nos métodos de teste, validação e resultados.
}
\palavraschave{Bitcoin, Redes neurais, previsão de preço.}


% --------------------------------------------------
% Keywords e abstract
% --------------------------------------------------
\abstract{%
The financial market has attracted attention for centuries, encompassing various assets such as rice, gold, and stocks. Price fluctuations can indicate significant events, and predicting these events can determine a business's success or failure. Over the years, various forecasting techniques, such as technical and fundamental analysis, have emerged. Recently, neural networks, especially Long Short-Term Memory (LSTM), have identified previously unnoticed patterns in time series data. Simultaneously, volatile markets like cryptocurrencies have emerged, particularly after the advent of Bitcoin and Blockchain technology in 2008, offering alternatives to combat inflation and centralized control. This study aims to analyze and compare algorithms using a real dataset to predict asset behavior in volatile markets, primarily in the context of Bitcoin. Different approaches based on recent findings were utilized to provide variability in testing methods, validation, and results.
}
\keywords{Bitcoin. Neural networks. Price prediction.}



% Dedicatória (opcional)
\textodedicatoria{%
A minha familia e amigos.

Principalmente a minha mãe, que sempre me apoiou e incentivou a seguir em frente.
}

% Agradecimentos (opcional)
\textoagradecimentos{%
Agradeço a todos que contribuíram para a realização deste trabalho.
}

% Epígrafe (opcional)
\textoepigrafe{%
$B > \frac{1}{n}\sum_{i=1}^{n}x_{i}$ \\
    
}

% Lista de siglas (opcional)
\listasiglas{%
 \begin{itemize}[]
  \item[BTC] -- \textit{Bitcoin}
  \item[FIAT] -- Moeda fiduciária
  \item[ARIMA] -- \textit{Autoregressive Integrated Moving Average}
  \item[ANN] -- \textit{Artificial Neural Network}
  \item[DNN] -- \textit{Deep Neural Network}
  \item[RNN] -- \textit{Recurrent Neural Network}
  \item[LSTM] -- \textit{Long Short-Term Memory}
  \item[BiLSTM] -- \textit{Bidirectional Long Short-Term Memory}
  \item[GRU] -- \textit{Gated Recurrent Unit}
  \item[MAE] -- \textit{Mean Absolute Error}
  \item[MASE] -- \textit{Mean Absolute Scaled Error}
  \item[RMSE] -- \textit{Root Mean Square Error}
  \item[MAPE] -- \textit{Mean Absolute Percentage Error}
  \item[$R^2$] -- \textit{R-squared}
 \end{itemize}
}
