% --------------------------------------------------
% Resumo e abstract (obrigatórios)
% --------------------------------------------------
\resumo{%
O mercado financeiro atrai a atenção de muitos há séculos, abrangendo ativos variados, como arroz, ouro e ações. A flutuação dos preços pode indicar eventos importantes, e prever tais eventos pode determinar o sucesso ou fracasso de um negócio. Diversas técnicas de previsão, como a análise técnica e fundamentalista, surgiram ao longo dos anos. Recentemente, redes neurais artificiais, especialmente as combinadas com células de memória \textit{Long Short-Term Memory} (LSTM), obtiveram sucesso na identificação de padrões em séries temporais antes despercebidos. 
Paralelamente, mercados voláteis como o de criptomoedas emergiram após o Bitcoin e a \textit{Blockchain} em 2008, oferecendo alternativas para combater a inflação e o controle centralizado. Este estudo visa analisar e comparar algoritmos, em uma base de dados real, que buscam predizer o comportamento dos ativos em mercados voláteis, principalmente no contexto do Bitcoin. 
Foram utilizadas diferentes abordagens baseadas em artigos recentes, obtendo-se um melhor desempenho em modelos estatísticos no intervalo de quinze minutos.
}
\palavraschave{Bitcoin, Redes neurais artificiais, previsão de preço.}


% --------------------------------------------------
% Keywords e abstract
% --------------------------------------------------
\abstract{%

The financial market has captured attention for centuries, including assets such as rice, gold, and stocks. Price fluctuations can reflect significant events, and the ability to predict these changes often determines the success or failure of businesses. Over the years, techniques like technical and fundamental analysis have been developed. 
Recently, artificial neural networks, particularly with the integration of Long Short-Term Memory (LSTM) cells, have gained prominence for identifying previously unnoticed patterns in time series data. 
Simultaneously, volatile markets like cryptocurrencies have emerged, driven by Bitcoin and Blockchain technology introduced in 2008. These markets offer alternatives to counter inflation and centralized control. This research aims to analyze and compare algorithms on a real-world dataset to predict asset behavior in volatile markets, focusing on Bitcoin. Based on recent research, the results reveal that statistical models achieved better performance for data sampled at fifteen-minute intervals.
}
\keywords{Bitcoin. Neural networks. Price prediction.}



% Dedicatória (opcional)
\textodedicatoria{%
À minha família e amigos.

Principalmente, à minha mãe, que sempre me apoiou e incentivou a seguir em frente.
}

% Agradecimentos (opcional)
\textoagradecimentos{%
Agradeço a todos que contribuíram para a realização deste trabalho.
}

% Epígrafe (opcional)
\textoepigrafe{%
$B > \frac{1}{n}\sum_{i=1}^{n}x_{i}$ \\
    
}

% Lista de siglas (opcional)
\listasiglas{%
 \begin{itemize}[]
  \item[BTC] -- \textit{Bitcoin}
  \item[ARIMA] -- \textit{Autoregressive Integrated Moving Average}
  \item[ANN] -- \textit{Artificial Neural Network}
  \item[DNN] -- \textit{Deep Neural Network}
  \item[RNN] -- \textit{Recurrent Neural Network}
  \item[LSTM] -- \textit{Long Short-Term Memory}
  \item[BiLSTM] -- \textit{Bidirectional Long Short-Term Memory}
  \item[GRU] -- \textit{Gated Recurrent Unit}
  \item[MAE] -- \textit{Mean Absolute Error}
  \item[MASE] -- \textit{Mean Absolute Scaled Error}
  \item[RMSE] -- \textit{Root Mean Square Error}
  \item[MAPE] -- \textit{Mean Absolute Percentage Error}
  \item[$R^2$] -- \textit{R-squared}
 \end{itemize}
}
