A história da matemática na predição de preços remonta ao trabalho pioneiro \textit{Théorie de la Spéculation} de \textcite{Bachelier} \cite{Courtault}. Fundamentado no conceito do movimento Browniano, 
fênomeno que descreve o deslocamento errático de partículas em um fluido, o autor o utiliza como analogia para o comportamento imprevisível dos preços de ativos no mercado financeiro. 
Essa ideia deu origem à teoria conhecida como \textit{Random Walk} (Caminhar Aleatório), que posteiormente serviu de base para outras modelagens, como a de \textcite{blacksholes}.

Dentre os ramos que compõem a projeção de preços, destacam-se as análises técnicas e fundamentalistas.
A análise técnica busca predizer os preços futuros baseadas em análises passadas de preço, volume e contratos abertos em opções \cite{Pring}.
Por outro lado, a análise fundamentalista se baseia em estimar o valor intrínseco dos ativos, ou seja, encontrar o preço justo por fundamentos do projeto, preços passados, situação atual e oportunidades futuras \cite{Ahmed}.

O \textit{Bitcoin} foi introduzido por \textcite{Nakamoto}, entidade desconhecida sob o pseudônimo de Satoshi Nakamoto. 
Considerada a primeira moeda digital implmementada que transaciona de maneira totalmente descentralizada, ou seja, sem a necessidade de uma autoridade central como um banco ou governo. 
Para isso, utiliza uma tecnologia denominada \textit{Blockchain}, que funciona como um livro caixa em uma estrutura de dados distribuída e imutável \cite{Ledger}. Desse modo, podem ser trocados por outras moedas ou ativos, como o Real, em plataformas de negociação de ativos digitais chamadas \textit{Exchanges}.
Existem diversass no mercado, cada uma com suas características e taxas, por isso, é preciso ter cuidado com a escolha da plataforma, pois existem algumas dessas que manipulam dados de transações \cite{FakeExchanges}.

A troca desses ativos gera um histórico de transações como em qualquer mercado financeiro,
porém, comsideravelmente mais volátil devido a falta de lastro das moedas FIAT tradicionais. 
Tendo em vista o caminhar aleatório do mercado, a análise preditiva não deve ser considerada uma tarefa trivial, mas, pode ser facilitada com o uso de técnicas de aprendizado de máquina.
Nos dias atuais, segundo \textcite{Fang}, os analistas de mercado têm feito uso não somente destes métodos como também de outros modelos computacionais convecionais ou híbridos, que têm por objetivo prever os preços, obtendo sucesso especialmente no contexto das criptomoedas \cite{Atsalakis}.

\section{Justificativa}
As redes neurais se tornaram uma peça fundamental na sociedade moderna,
por isso, novas técnicas surgem a cada dia para resolver múltiplos problemas, 
muitas vezes não solucionáveis por humanos.
Por outro lado, o mercado de ativos baseados em criptografia cresce exponencialmente, 
sendo muito utilizado como reserva de valor, meio de pagamento e uma forma de diversificação de investimentos.

Então, justifica-se esse trabalho pois essas duas tecnologias convergem
em um mundo cada vez mais digital, a união das mesmas pode até mesmo ser intrínseca a muitas criptomoedas.
A previsão de preços em mercados não é nova, mas, em ativos voláteis como o Bitcoin sempre existem oportunidades de operações lucrativas. 
Com base nesses termos, esse projeto visa trazer à tona dois temas muito relevantes atualmente, principalmente em áreas que estão sob a ótica da engenharia de computação.


\section{Proposta}

Este estudo tem como proposta analisar e avaliar algoritmos voltados para previsão de preços, espera-se verificar sua lucratividade em operações usando um dos mais famosos criptoativos, o Bitcoin.
Os modelos serão avaliados de acordo com suas previsões em um cenário real de variação de preço, estruturados por meio de uma série temporal histórica. A pesquisa utilizará essa mesma base de dados e manterá os métodos de pré-processamento fixos, variando no contexto da aplicação apenas o algoritmo de previsão.
Isto será feito de forma a manter as condições de experimentação para todos os algoritmos, garantindo assim que a diferença esteja apenas no funcionamento de cada um.
Os dados serão obtidos através dos registros de negociações reais na \textit{Exchange}
 Binance, que contém informações como o preço, volume e o número de transações realizadas entre janeiro a setembro de 2020.
As arquiteturas a serem exploradas incluem a ARIMA, LSTM, BiLSTM e GRU.

\subsection{Objetivo geral}

Analisar por meio comparativo o desempenho de algoritmos de predição de preço no contexto do Bitcoin.

\subsection{Objetivos específicos}

Para atingir o objetivo principal, se fazem necessários os seguintes objetivos específicos:
\begin{itemize}
    \item Desenvolvimento da estrutura computacional necessária para selecionar, implementar e realizar previsões por meio de ferramentas tecnológicas adequadas (como linguagens de programação, métodos de extração e armazenamento de dados, geradores gráficos e demais ferramentas que se fizerem necessárias);
    \item Avaliação de algoritmos de redes neurais e compará-los, frente aos \textit{Benchmarks} de interesse, a fim de determinar qual tem melhor desempenho;
    \item Exploração de possíveis variações em métodos conhecidos, visando adaptá-los a um novo cenário;
    \item Análise se esses métodos de predição são rentáveis em uma base de dados real.
\end{itemize}

\section{Estrutura do documento}

Este trabalho está estruturado em seis capítulos que incluindo também as referências. 

    Após a introdução, são apresentados os fundamentos teóricos necessários para a compreensão das principais técnicas empregadas na metodologia. 
Desse modo, o capítulo \ref{cap:fundamentacao_teorica} aborda conceitos de criptomoedas e séries temporais, fornecendo uma introdução à análise de dados, explorando artigos relacionados 
à estrutura do mercado e a predição de preços. 
Além disso, é apresentado o estado da arte em relação a redes neurais e suas aplicações no contexto do mercado financeiro.

O Capítulo \ref{cap:metodologia} apresenta a metodologia, organizada de acordo com a sequência de implementação da solução.

    No capítulo \ref{cap:resultados}, são descritos os resultados e discussões dos experimentos realizados.

    O capítulo \ref{cap:conclusao} oferece uma conclusão que abrange as principais contribuições, limitações da pesquisa e propostas para trabalhos futuros.